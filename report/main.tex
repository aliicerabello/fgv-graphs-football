\documentclass[12pt, a4paper]{report} % Tamanho padrão de relatório
\usepackage[utf8]{inputenc}
\usepackage[brazil]{babel}
\usepackage{geometry}
\geometry{a4paper, left=3cm, right=2cm, top=3cm, bottom=2cm} % Margens adequadas

% Pacotes essenciais para formatação profissional
\usepackage{setspace}
\onehalfspacing % Espaçamento 1.5 - padrão acadêmico

% NO PREÂMBULO
\usepackage{etoolbox}
\makeatletter
\patchcmd{\chapter}{\if@openright\cleardoublepage\else\clearpage\fi}{%
    \if@openright\cleardoublepage\else\clearpage\fi\vspace*{-50pt}}{}{}
\makeatother

\usepackage{indentfirst} % Recuo na primeira linha do parágrafo
\setlength{\parindent}{1.25cm} % Recuo de 1,25cm

\usepackage{titlesec}
\titlespacing*{\chapter}{0pt}{10pt}{15pt} % {esq}{acima}{abaixo}

\usepackage[T1]{fontenc}
\usepackage{amsmath, amssymb, graphicx, booktabs, array, hyperref, listings, xcolor, float, tabularx}

% --- Definição de Estilo do Código ---
\definecolor{codegray}{rgb}{0.5,0.5,0.5}
\definecolor{codepurple}{rgb}{0.58,0,0.82}
\definecolor{backcolour}{rgb}{0.98,0.98,0.98}

\lstdefinestyle{PythonStyle}{
    backgroundcolor=\color{backcolour},   
    commentstyle=\color{codegray},
    keywordstyle=\color{blue},
    numberstyle=\tiny\color{codegray},
    stringstyle=\color{codepurple},
    basicstyle=\ttfamily\footnotesize, 
    breakatwhitespace=false,         
    breaklines=true,                 
    captionpos=b,                    
    keepspaces=true,                 
    numbers=left,                    
    numbersep=5pt,                  
    showspaces=false,                
    showstringspaces=false,
    showtabs=false,                  
    tabsize=2,
    language=Python,
    extendedchars=true,
    inputencoding=utf8, 
    literate={ã}{{\~a}}1 {é}{{\'e}}1 {ç}{{\c c}}1 {í}{{\'i}}1 {á}{{\'a}}1 {ú}{{\'u}}1 {õ}{{\~o}}1 {ó}{{\'o}}1
}
\lstset{style=PythonStyle}
% -------------------------------------

\title{Análise Computacional de Estratégias Coletivas no Futebol: \\
Uma Aplicação da Teoria dos Grafos a Dados Reais de Partidas}
\author{Alice Rabello Oliveira \\ Pablo Levy Fernandes Alcântara \\ Raul Medici Martinelli}
\date{\today}

\begin{document}

\maketitle
\thispagestyle{empty} 

\newpage

\chapter*{Resumo}

Este trabalho aplica conceitos de Teoria dos Grafos para análise computacional de estratégias coletivas no futebol, utilizando dados reais de partidas da plataforma \textit{StatsBomb}. Desenvolvemos uma metodologia geral e modular que representa jogadores como vértices e passes decisivos como arestas direcionadas e ponderadas, permitindo a identificação de padrões táticos e interações relevantes entre atletas. A abordagem considera passes que resultam em finalizações, assistências, contra-ataques e avanços significativos em direção ao gol adversário, com pesos das arestas indicando a frequência ou importância dessas conexões.

Como estudo de caso, analisamos a partida final da Copa do Mundo de 2022 entre Argentina e França, complementada por uma segunda partida para validação da generalidade do método. A implementação em Python, utilizando bibliotecas como \texttt{statsbombpy}, \texttt{pandas}, \texttt{networkx} e \texttt{matplotlib}, permitiu a geração de grafos ponderados, matrizes de adjacência e visualizações das redes de passes.

Os resultados obtidos incluem a identificação de jogadores-chave com maior centralidade nas trocas de bola, reconhecimento de grupos de colaboração frequente e caracterização dos \textit{matchings} mais relevantes entre atacantes e defensores das equipes adversárias. A análise demonstra como as interações entre atletas formam sistemas coletivos complexos, proporcionando insights quantitativos sobre dinâmicas táticas que frequentemente passam despercebidas na análise convencional.

\newpage 

\chapter*{Introdução}
\addcontentsline{toc}{chapter}{Introdução}
A Teoria dos Grafos tem se mostrado uma ferramenta poderosa para compreender a complexidade tática do futebol moderno. Representar passes, pressões e confrontos diretos como grafos possibilita quantificar relações invisíveis à observação humana e apoiar decisões estratégicas com base em dados.

Este trabalho emprega um modelo computacional baseado em Teoria dos Grafos para identificar \textit{nós} (jogadores) e \textit{arestas} (passes decisivos) que estruturam redes coletivas no futebol. Como casos de estudo, analisamos dois jogos da Copa de 2022 com características táticas distintas: a final entre a Argentina e França, considerado um dos maiores espetáculos táticos ofensivos da década; e o jogo entre Marrocos e Espanha, onde uma equipe defensiva neutralizou a posse de bola dominante adversária.

A metodologia desenvolvida é escalável e replicável para qualquer partida da base StatsBomb, permitindo transformar dados brutos em inteligência tática acionável através da quantificação de interações em campo \cite{statsperform}. A abordagem demonstra como modelos matemáticos podem revelar padrões coletivos invisíveis à observação convencional, contribuindo para a evolução do futebol de arte empírica para ciência baseada em dados.

\textbf{A abordagem desenvolvida demonstra como conceitos matemáticos abstratos podem ser aplicados para iluminar aspectos concretos do jogo, oferecendo uma ferramenta objetiva para análise tática tanto no âmbito acadêmico quanto profissional.}


O código completo e os arquivos de análise estão disponíveis em repositório público no GitHub: \url{https://github.com/pablolevy/futebol-grafos-2022}

\section*{Objetivos}
\begin{itemize}
    \item Aplicar a Teoria dos Grafos para representar jogadores como vértices e passes decisivos como arestas direcionadas e ponderadas.
    \item Identificar as estruturas ofensivas e defensivas de cada equipe.
    \item Reconhecer grupos de colaboração frequente e jogadores com maior envolvimento nas trocas de bola.
    \item Caracterizar os \textit{matchings} mais relevantes entre jogadores adversários.
\end{itemize}

\chapter{Fundamentação Teórica: Teoria dos Grafos Aplicada ao Futebol}

\section{Grafos Dirigidos e Ponderados}
Um grafo dirigido (ou dígrafo) $G = (V, E)$ é composto por um conjunto de vértices ($V$) e um conjunto de arestas direcionadas ($E$). Em nosso modelo, $V$ representa os jogadores e $E$ representa os passes decisivos entre eles. O grafo é \textbf{ponderado} ($G = (V, E, w)$), onde $w(u,v)$ indica o peso da aresta do vértice $u$ para $v$, representando a frequência ou importância dos passes decisivos do jogador $u$ para o jogador $v$.

\section{Matriz de Adjacência e Graus de Vértices}

A matriz de adjacência $A$ de um grafo com $n$ vértices é uma matriz $n \times n$ onde o elemento $A_{ij}$ representa o peso da aresta do vértice $i$ para o vértice $j$. 

A análise das matrizes de adjacência baseia-se no cálculo do Grau de Entrada e Saída.
O \textbf{Grau de Saída} ($\text{out-degree}$) e o \textbf{Grau de Entrada} ($\text{in-degree}$) são calculados como:
$$
\text{out-degree}(u) = \sum_{v \in V} A_{uv} \quad \text{e} \quad \text{in-degree}(v) = \sum_{u \in V} A_{uv}
$$

O Grau de Saída indica a capacidade de distribuição de um jogador, enquanto o Grau de Entrada mede o quanto ele é buscado como alvo na progressão da jogada.

\section{Grafos Bipartidos e Análise de Matchings Ponderados}

Além da análise estrutural das redes de passes, a Teoria dos Grafos oferece ferramentas para estudar interações específicas entre subconjuntos de jogadores, particularmente através do conceito de \textit{matchings} em grafos bipartidos.

O conceito de \textit{matching} (ou emparelhamento) é uma das estruturas fundamentais da Teoria dos Grafos e tem aplicação direta na análise de confrontos individuais em esportes coletivos. Em um \textbf{grafo bipartido} $G = (U \cup V, E)$, os vértices são divididos em dois conjuntos disjuntos $U$ e $V$, de modo que cada aresta conecta um vértice de $U$ a um vértice de $V$. No contexto deste trabalho, $U$ representa os \textbf{jogadores ofensivos} de uma equipe, enquanto $V$ representa os \textbf{defensores} da equipe adversária. Cada aresta $(u, v) \in E$ indica que o atacante $u$ foi diretamente confrontado pelo defensor $v$ em uma ação de jogo (como desarme, bloqueio, interceptação ou falta sofrida).

Um \textbf{matching} $M \subseteq E$ é um conjunto de arestas sem vértices em comum — isto é, cada jogador participa de no máximo um confronto dentro desse conjunto. Essa estrutura representa o conjunto de duelos exclusivos que podem ocorrer simultaneamente entre os dois conjuntos de jogadores.

O modelo computacional implementado neste trabalho utiliza a função \texttt{max\_weight\_matching} da biblioteca \texttt{NetworkX}, que calcula o \textbf{matching máximo ponderado}. Nesse caso, o algoritmo busca o subconjunto $M \subseteq E$ que \textbf{maximiza a soma dos pesos das arestas}, onde o peso $w(u,v)$ indica a intensidade do confronto entre o atacante $u$ e o defensor $v$. Assim, o matching retornado não é necessariamente o que contém o maior número de duelos (\textit{matching máximo cardinal}), mas sim aquele que concentra os duelos de \textbf{maior relevância tática}.

\begin{equation}
M^* = \arg\max_{M \subseteq E} \sum_{(u,v) \in M} w(u,v)
\end{equation}

No contexto do futebol, essa formulação permite identificar os \textbf{duelos de maior impacto} em uma partida, priorizando as interações mais intensas — como desarmes decisivos, bloqueios de finalizações e interceptações em zonas críticas. Cada peso $w(u,v)$ é calculado com base no tipo e na importância da ação defensiva (por exemplo, 3 pontos para um desarme vitorioso, 2 para um bloqueio e 1 para uma interceptação), o que confere significado tático direto à métrica.

A análise do \textit{matching máximo ponderado} fornece informações sobre:
\begin{itemize}
    \item \textbf{Eficiência defensiva}: defensores com múltiplas arestas de alto peso indicam sobrecarga ou foco de marcação;
    \item \textbf{Direcionamento ofensivo}: atacantes com alto peso acumulado refletem a concentração das ações ofensivas;
    \item \textbf{Equilíbrio entre setores}: um matching mais distribuído revela uma equipe que varia os pontos de ataque, enquanto um matching concentrado indica dependência tática de um setor ou jogador específico.
\end{itemize}

Em termos táticos, o \textbf{matching máximo ponderado} identifica, entre todos os confrontos possíveis, aqueles de maior intensidade acumulada — isto é, os embates que mais influenciaram o comportamento coletivo das equipes.

\chapter{Metodologia}

\section{Ferramentas e Bibliotecas Utilizadas}
A análise foi inteiramente desenvolvida em Python, utilizando um conjunto de bibliotecas específicas para manipulação de dados de futebol e modelagem de grafos:
\begin{itemize}
    \item \texttt{statsbombpy}: Biblioteca de código aberto fundamental para a \textbf{coleta e acesso aos dados brutos} (eventos de passes, chutes, etc.) da plataforma StatsBomb.
    \item \texttt{pandas}: Utilizada para a \textbf{organização, processamento e filtragem} dos dados brutos em DataFrames, culminando na criação das matrizes de adjacência (CSV).
    \item \texttt{networkx}: Biblioteca central para a \textbf{modelagem do grafo}. Ela permite a criação dos objetos \texttt{DiGraph} (Redes de Passes) e \texttt{Graph} (Matchings) e o cálculo de propriedades como os graus de vértice.
    \item \texttt{matplotlib}: Utilizada para a \textbf{visualização} dos grafos, permitindo que as arestas e nós sejam renderizados de forma proporcional ao seu peso e grau, respectivamente.
\end{itemize}

\section{Definição e Filtragem de Passes Decisivos}

A determinação de um passe como "decisivo"  é o passo crucial para construir a rede. Nossas arestas representam apenas as interações que avançam o sistema ofensivo em direção a uma ameaça de gol.

\subsection{Critérios e Parâmetros de Filtragem}
O código utiliza a função \texttt{identificar\_passes\_decisivos} para aplicar um conjunto de critérios estritos sobre os eventos de passe. Um passe é considerado decisivo se atender a \textbf{pelo menos uma} das seguintes condições, garantindo que a interação seja de alto valor tático:
\begin{enumerate}
    \item \textbf{Criação de Gol ou Finalização} (\texttt{pass\_goal\_assist} ou \texttt{pass\_shot\_assist}).
    \item \textbf{Quebra de Linhas} (\texttt{pass\_through\_ball}).
    \item \textbf{Ataque pela Lateral} (\texttt{pass\_cross}).
    \item \textbf{Transição Rápida} (\texttt{play\_pattern == "Counter Attack"}).
\end{enumerate}
Uma \textbf{condição obrigatória} é que o passe deve ter um destinatário identificado \\
(\texttt{pd.notna(passe.get("pass\_recipient"))}).

\begin{figure}[H]
    \centering
    \begin{lstlisting}[caption={Trecho do Código: Função identificar\_passes\_decisivos (passes.py)}]
def identificar_passes_decisivos(passes_df, team_name, all_events):
    decisivos = []
    
    for _, passe in passes_df.iterrows():
        try:
            condicoes = [
                passe.get("pass_goal_assist"),     # 1. Assistencia para gol
                passe.get("pass_shot_assist"),     # 2. Assistencia para chute
                passe.get("pass_through_ball"),    # 3. Bola enfiada
                passe.get("pass_cross"),           # 4. Cruzamento
                passe.get("play_pattern") == "Counter Attack", # 5. Contra-ataque
            ]
            if any(condicoes) and pd.notna(passe.get("pass_recipient")):
                decisivos.append(
                    {
                        "player": passe["player"],
                        "pass_recipient": passe["pass_recipient"],
                        "team": team_name,
                    }
                )
        except (KeyError, TypeError):
            continue

    return decisivos
\end{lstlisting}
    \caption{Lógica de Filtragem da Função \texttt{identificar\_passes\_decisivos}}
    \label{fig:parametros_passes}
\end{figure}

\section{Modelagem do Grafo}
\subsection{Rede de Passes Decisivos}
O código a seguir demonstra a criação do grafo dirigido e ponderado a partir da matriz de adjacência:

\begin{lstlisting}[caption={Trecho do Código: Criação e Ponderação do Grafo de Passes}]
import pandas as pd
import networkx as nx

# 1. Carregamento da Matriz de Adjacencia
df_passes = pd.read_csv("matriz_passes_Argentina_3869685.csv", index_col=0)
G_passes = nx.DiGraph()

# 2. Inclusao dos nos (jogadores)
G_passes.add_nodes_from(df_passes.index)

# 3. Adicao das arestas ponderadas (frequencia de passes)
for i in df_passes.index:
    for j in df_passes.columns:
        weight = df_passes.loc[i, j]
        if weight > 0:
            G_passes.add_edge(i, j, weight=weight)
\end{lstlisting}

\subsection{Confrontos Diretos (\textit{Matchings})}
Os confrontos são modelados como um grafo bipartido, utilizando a matriz de confrontos:

\begin{lstlisting}[caption={Trecho do Código: Criação do Grafo Bipartido de Matchings}]
# 1. Carregamento da Matriz de Matchings
df_matchings = pd.read_csv("matriz_defensores_Argentina_France_3869685.csv", index_col=0)
G_match = nx.Graph()

# 2. Definicao dos dois conjuntos de nos (Equipe A e Equipe B)
nodes_a = list(df_matchings.index)
nodes_b = list(df_matchings.columns)

# 3. Adicao das arestas ponderadas (frequencia de confrontos)
for player_a in nodes_a:
    for player_b in nodes_b:
        weight = df_matchings.loc[player_a, player_b]
        if weight > 0:
            G_match.add_edge(player_a, player_b, weight=weight)
\end{lstlisting}

\chapter{Resultados Computacionais}

\section{Análise do Grau de Vértices (Passes Decisivos)}
A análise dos graus é fundamental para identificar a importância posicional de cada jogador dentro da rede.

\subsection{Argentina vs França: Contraste de Estilos}
\subsubsection{Argentina: O Foco no Centro}
A Tabela \ref{tab:graus_arg} demonstra uma clara concentração da distribuição de passes no meio-campo. \textbf{Rodrigo De Paul} e \textbf{Enzo Fernandez} são os principais distribuidores.

\begin{table}[H]
    \centering
    \caption{Graus de Entrada e Saída - Argentina}
    \label{tab:graus_arg}
    \begin{tabularx}{\textwidth}{lXXXc}
    \toprule
    \textbf{Jogador} & \textbf{Grau de Saída} & \textbf{Grau de Entrada} & \textbf{Diferença} & \textbf{Papel} \\
    \midrule
    Rodrigo Javier De Paul & 68 & 44 & +24 & \textbf{Distribuidor Primário} \\
    Enzo Fernandez & 67 & 47 & +20 & Pivô Central \\
    Lionel Messi & 44 & 43 & +1 & Conector Tático \\
    \bottomrule
    \end{tabularx}
\end{table}

\subsubsection{França: A Construção Defensiva}
Na Tabela \ref{tab:graus_fra}, o destaque fica para a linha defensiva como iniciadora das jogadas.

\begin{table}[H]
    \centering
    \caption{Graus de Entrada e Saída - França}
    \label{tab:graus_fra}
    \begin{tabularx}{\textwidth}{lXXXc}
    \toprule
    \textbf{Jogador} & \textbf{Grau de Saída} & \textbf{Grau de Entrada} & \textbf{Diferença} & \textbf{Papel} \\
    \midrule
    Raphaël Varane & 60 & 37 & +23 & \textbf{Iniciador Defensivo} \\
    Jules Koundé & 51 & 38 & +13 & Conector Lateral \\
    Kylian Mbappé & 20 & 26 & -6 & \textbf{Alvo de Ataque} \\
    \bottomrule
    \end{tabularx}
\end{table}

\subsection{Marrocos vs Espanha: Eficiência vs Posse}
\subsubsection{Espanha: Domínio Centralizado}
A Tabela \ref{tab:graus_esp} revela a extrema centralização no meio-campo espanhol.

\begin{table}[H]
    \centering
    \caption{Graus de Entrada e Saída - Espanha}
    \label{tab:graus_esp}
    \begin{tabularx}{\textwidth}{lXXXc}
    \toprule
    \textbf{Jogador} & \textbf{Grau de Saída} & \textbf{Grau de Entrada} & \textbf{Diferença} & \textbf{Papel} \\
    \midrule
    Rodrigo Hernández & 399 & 210 & +189 & \textbf{Pivô Absoluto} \\
    Sergio Busquets & 245 & 198 & +47 & Organizador \\
    Jordi Alba & 187 & 145 & +42 & Lateral Conector \\
    \bottomrule
    \end{tabularx}
\end{table}

\subsubsection{Marrocos: Eficiência nas Transições}
A Tabela \ref{tab:graus_mar} mostra uma rede mais equilibrada e eficiente.

\begin{table}[H]
    \centering
    \caption{Graus de Entrada e Saída - Marrocos}
    \label{tab:graus_mar}
    \begin{tabularx}{\textwidth}{lXXXc}
    \toprule
    \textbf{Jogador} & \textbf{Grau de Saída} & \textbf{Grau de Entrada} & \textbf{Diferença} & \textbf{Papel} \\
    \midrule
    Azzedine Ounahi & 78 & 45 & +33 & \textbf{Organizador} \\
    Sofyan Amrabat & 65 & 52 & +13 & Pivô Defensivo \\
    Hakim Ziyech & 58 & 49 & +9 & Criador \\
    \bottomrule
    \end{tabularx}
\end{table}

\section{Visualização das Redes de Colaboração}

\subsection{Argentina vs França}
\begin{figure}[H]
    \centering
    \begin{minipage}{0.48\textwidth}
        \centering
        \includegraphics[width=\textwidth]{grafo_passes_Argentina_3869685.png}
        \caption{Rede de Passes - Argentina}
    \end{minipage}\hfill
    \begin{minipage}{0.48\textwidth}
        \centering
        \includegraphics[width=\textwidth]{grafo_passes_France_3869685.png}
        \caption{Rede de Passes - França}
    \end{minipage}
    \label{fig:grafos_arg_fra}
\end{figure}

\subsection{Marrocos vs Espanha}
\begin{figure}[H]
    \centering
    \begin{minipage}{0.48\textwidth}
        \centering
        \includegraphics[width=\textwidth]{grafo_passes_Morocco_3869220.png}
        \caption{Rede de Passes - Marrocos}
    \end{minipage}\hfill
    \begin{minipage}{0.48\textwidth}
        \centering
        \includegraphics[width=\textwidth]{grafo_passes_Spain_3869220.png}
        \caption{Rede de Passes - Espanha}
    \end{minipage}
    \label{fig:grafos_mar_esp}
\end{figure}

\section{Análise dos Confrontos Diretos (\textit{Matchings})}

\begin{figure}[H]
\centering
\begin{minipage}{0.48\textwidth}
\centering
\includegraphics[width=0.85\linewidth, height=6cm]{grafo_matching_Argentina_France_3869685.png}
\caption{Matchings: Argentina (Ofensiva) vs. França (Defesa)}
\end{minipage}\hfill
\begin{minipage}{0.48\textwidth}
\centering
\includegraphics[width=0.85\linewidth, height=6cm]{grafo_matching_France_Argentina_3869685.png}
\caption{Matchings: França (Ofensiva) vs. Argentina (Defesa)}
\end{minipage}
\end{figure}

\begin{figure}[H]
\centering
\begin{minipage}{0.48\textwidth}
\centering
\includegraphics[width=0.85\linewidth, height=6cm]{grafo_matching_Morocco_Spain_3869220.png}
\caption{Matchings: Marrocos (Ofensiva) vs. Espanha (Defesa)}
\end{minipage}\hfill
\begin{minipage}{0.48\textwidth}
\centering
\includegraphics[width=0.85\linewidth, height=6cm]{grafo_matching_Spain_Morocco_3869220.png}
\caption{Matchings: Espanha (Ofensiva) vs. Marrocos (Defesa)}
\end{minipage}
\end{figure}

\vspace{-0.3cm}
\begin{itemize}\setlength\itemsep{0.1em}
    \item \textbf{Argentina × França}: Messi $\leftrightarrow$ Camavinga (maior peso)
    \item \textbf{Marrocos × Espanha}: Boufal $\leftrightarrow$ Busquets, Hakimi $\leftrightarrow$ Carvajal, Torres $\leftrightarrow$ Mazraoui
\end{itemize}

\chapter{Discussão da Análise Tática e Corroboração}


\section{Argentina: o motor central de De Paul e Enzo Fernández}
Os vértices de maior grau na rede argentina foram Rodrigo De Paul e Enzo Fernández, indicando que o meio-campo atuou como principal canal de progressão ofensiva. Essa estrutura corrobora a análise do portal \textit{Coaches’ Voice}, que apontou que “the control of De Paul, Fernández and Mac Allister in midfield provided Argentina with a platform to dictate tempo and dominate possession” \cite{coachesvoice}. 

Visualmente, o grafo de passes decisivos mostra uma rede compacta e centralizada, com Messi ocupando papel de conector, recebendo e redistribuindo passes entre o meio e o ataque — padrão também reconhecido pelo relatório técnico da FIFA \cite{fifareport}.

\section{França: construção sob pressão e o papel de Varane}
Na rede francesa, Raphaël Varane e Upamecano apresentaram os maiores graus de saída, confirmando que a França tentou construir a partir da defesa mesmo sob forte pressão. O relatório técnico da FIFA \cite{fifareport} e a análise publicada por \textit{The Mastermind Site} \cite{mastermind} observam que a Argentina aplicou uma pressão coordenada que “forced France’s centre-backs into recycling possession and reduced their passing lanes”. 

O modelo de grafo reflete esse comportamento: passes densos entre os zagueiros e escassez de conexões verticais até o meio-campo.

\section{Marrocos: eficiência defensiva e transições por Ounahi}
A rede marroquina apresentou Azzedine Ounahi como vértice central, porém com volume significativamente menor (78 conexões) em comparação com o pivô espanhol Rodri (399 conexões). Esta disparidade reflete a estratégia de transições rápidas em contra-ataque, corroborada pela análise de Nanavati (2022) \cite{nanavati2022}, que destaca: "\textit{Morocco's ability to quickly transition from defense to attack through Ounahi and Ziyech was key to breaking Spain's press}".

O grafo de passes mostra uma rede mais esparsa mas eficiente, com conexões estratégicas diretas entre defesa e ataque, evitando a circulação lateral característica da Espanha.

\section{Espanha: posse estéril e centralização extrema em Rodri}
A análise revelou uma centralização extrema no meio-campo espanhol, com Rodri apresentando 399 conexões - cinco vezes mais que o principal organizador marroquino. Esta dependência excessiva corrobora a observação de Nanavati (2022) \cite{nanavati2022} sobre "\textit{Spain's sterile possession and lack of vertical progression}".

O modelo de grafo demonstra claramente o "efeito funil" da construção espanhola, onde praticamente todas as jogadas passavam por Rodri, criando um ponto único de falha que foi explorado pelo pressing marroquino.

\section{Análise Comparativa dos Matchings Estratégicos}

\subsection{O duelo-chave: Messi vs. Camavinga}

Na final analisada, o matching de maior peso foi o duelo \textbf{Messi $\leftrightarrow$ Camavinga}, o que reflete a decisão do técnico francês Didier Deschamps de posicionar Camavinga como lateral-esquerdo com a função de acompanhar Messi entre as linhas. Essa observação é corroborada pela análise do \textit{Citizen Digital} (2022), que destaca que “\textit{Camavinga, being a natural midfielder, offered flexibility, as he could use his skill set to track Messi’s every move and create overload down midfield}” \cite{citizen_camavinga}.

\subsection{Mbappé como vértice terminal}
Os dados revelam que Mbappé teve um Grau de Entrada superior ao de Saída, indicando que era o principal destino dos passes decisivos — o vértice terminal da rede ofensiva francesa. Essa conclusão é suportada pelas estatísticas do \textit{WhoScored} \cite{whoscored}, que relatam que “Mbappé had the highest xG of the match (1.9) and received the majority of France’s progressive passes”. 
O banco de dados \textit{FBref}, também baseado em StatsBomb, confirma que ele foi o jogador mais acionado nas zonas de finalização \cite{fbref}.

\subsection{Matchings estratégicos: o pressing organizado marroquino}

No confronto Marrocos x Espanha, os matchings de maior peso identificados foram \textbf{Sofiane Boufal $\leftrightarrow$ Sergio Busquets}, \textbf{Achraf Hakimi $\leftrightarrow$ Dani Carvajal}, e \textbf{Ferran Torres $\leftrightarrow$ Noussair Mazraoui}, demonstrando a estratégia marroquina de pressionar pontos-chave em múltiplas frentes. Estes matchings corroboram análises especializadas que destacaram a eficácia da estratégia marroquina. McNulty (2022) \cite{bbc2022} observou que "\textit{Morocco pressed Spain high, forcing them into mistakes}", corroborando o matching \textbf{Boufal $\leftrightarrow$ Busquets}, enquanto Kennedy (2022) \cite{npr2022} destacou o "\textit{disciplined defensive structure}" que frustrou os atacantes espanhóis, refletindo nos matchings \textbf{Hakimi $\leftrightarrow$ Carvajal} e \textbf{Torres $\leftrightarrow$ Mazraoui}.
Essa observação reforça que o defensor foi posicionado estrategicamente para seguir Messi entre as linhas e gerar superioridade numérica no meio-campo — comportamento refletido na intensidade de confrontos identificada no grafo.

\section{O futuro da análise de dados no futebol}
O uso de modelos de rede e análise de grafos no futebol é uma tendência consolidada. Segundo a \textit{StatsPerform} \cite{statsperform}, “network-based models and passing graphs are becoming standard tools for elite tactical analysis”.

Nosso estudo demonstra como a aplicação prática desses modelos pode revelar estruturas de colaboração, zonas de pressão e duelos táticos com objetividade matemática.

\chapter{Conclusão}
A aplicação da Teoria dos Grafos às partidas da Copa do Mundo de 2022 mostrou-se
eficaz na identificação de padrões táticos e duelos estruturais em contextos distintos. 
Tanto no embate ofensivo entre Argentina e França quanto no confronto tático entre 
Marrocos e Espanha, as redes de passes e os \textit{matchings} revelaram interdependências 
entre setores e confirmaram achados qualitativos observados por analistas esportivos.

O modelo computacional desenvolvido demonstrou versatilidade ao capturar desde
a centralização extrema da Espanha até a eficiência transicional de Marrocos, 
validando-se como ferramenta replicável para análise de sistemas coletivos sob 
diferentes filosofias táticas. Esta abordagem consolida a ponte entre a Matemática 
Discreta e as ciências do esporte, oferecendo um meio objetivo para decifrar a 
complexidade do futebol moderno.
\begin{thebibliography}{99}

\bibitem{fifareport}
FIFA Technical Study Group. \textit{Argentina 3–3 France — World Cup Final — Match Analysis (Technical Report)}. FIFA Training Centre, 2022.
Disponível em: \url{https://www.fifatrainingcentre.com/media/native/world-cup-2022/report_128083.pdf}. Acesso em: 11 nov. 2025.

\bibitem{coachesvoice}
Coaches’ Voice. \textit{World Cup final 2022 tactics: Argentina v France}. 19 dez. 2022.
Disponível em: \url{https://learning.coachesvoice.com/cv/world-cup-final-2022-tactics-argentina-messi-france-mbappe/}. Acesso em: 11 nov. 2025.

\bibitem{mastermind}
Desmond, Rhys. \textit{Argentina 3–3 France – World Cup Final – Match Analysis}. The Mastermind Site, 19 dez. 2022.
Disponível em: \url{https://themastermindsite.com/2022/12/19/argentina-3-3-france-world-cup-final-match-analysis/}. Acesso em: 11 nov. 2025.

\bibitem{citizen_camavinga}
Citizen Digital. \textit{Qatar World Cup 2022 Final – The Talking Points}. 19 dez. 2022.
Disponível em: \url{https://www.citizen.digital/article/qatar-world-cup-2022-final-the-talking-points-n311471}. Acesso em: 11 nov. 2025.

\bibitem{espn}
Sagar, Srinivas. \textit{2022 World Cup final: Decoding the tactical battle between Argentina and France}. ESPN, 17 dez. 2022.
Disponível em: \url{https://www.espn.co.uk/football/story/_/id/35268409/2022-world-cup-final-argentina-messi-france-mbappe-tactical-battle}. Acesso em: 11 nov. 2025.

\bibitem{whoscored}
WhoScored. \textit{Argentina 3–3 France — Match Report (World Cup Final)}. 2022.
Disponível em: \url{https://www.whoscored.com/Matches/1611851/Live/World-Cup-2022-Argentina-France}. Acesso em: 11 nov. 2025.

\bibitem{fbref}
FBref. \textit{Kylian Mbappé — Player match logs and statistics}. 2022–2023.
Disponível em: \url{https://fbref.com/en/players/42fd9c7f/Kylian-Mbappe}. Acesso em: 11 nov. 2025.

\bibitem{statsperform}
StatsPerform. \textit{The Future of Football Analytics: Network Models and Tactical Insights}. Technical Report, 2023.
Disponível em: \url{https://www.statsperform.com/resource/the-future-of-football-analytics}. Acesso em: 11 nov. 2025.

\bibitem{bbc2022}
McNulty, Phil. \textit{World Cup 2022: Morocco 0-0 Spain}. BBC Sport, 6 dez. 2022.
Disponível em: \url{https://www.bbc.com/sport/football/63789751}. Acesso em: 11 nov. 2025.

\bibitem{npr2022}
Kennedy, Merrit. \textit{Morocco beats Spain in a stunning World Cup upset}. NPR, 6 dez. 2022.
Disponível em: \url{https://www.npr.org/2022/12/06/1140967981/morocco-spain-world-cup-upset-penalties-yassine-bounou}. Acesso em: 11 nov. 2025.

\bibitem{nanavati2022}
Nanavati, Vyom. \textit{Game Plan: FIFA World Cup 2022 — Morocco v Spain — Breakdown of Key Tactics}. Medium, 6 dez. 2022.
Disponível em: \url{https://medium.com/@vyom.nanavati/game-plan-fifa-world-cup-2022-morocco-v-spain-breakdown-of-key-tactics-ahead-of-key-clash-4507b3887958}. Acesso em: 11 nov. 2025.

\end{thebibliography}

\end{document}
