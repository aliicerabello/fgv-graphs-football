\documentclass[12pt, a4paper]{report}
\usepackage[utf8]{inputenc}
\usepackage[brazil]{babel}
\usepackage{geometry}
\geometry{a4paper, left=2.5cm, right=2cm, top=1cm, bottom=1.5cm}

% Pacotes essenciais
\usepackage{setspace}
\onehalfspacing

\usepackage{indentfirst}
\setlength{\parindent}{1.25cm}

\usepackage{titlesec}

% ---- Configuração do Sumário (TOC) ----
% Mostra apenas até o nível de Seção (Capítulo e Seção), escondendo subseções.
\setcounter{tocdepth}{1} 

% ---- Formatação de capítulo sem "Capítulo" ----
\titleformat{\chapter}
{\normalfont\Large\bfseries}
{\thechapter\ }{0pt}{}
\titlespacing*{\chapter}{0pt}{10pt}{8pt}

% ====================================================================
% CORREÇÃO: USAR COMANDOS MAIS SIMPLES PARA EVITAR QUEBRAS DE PÁGINA
% ====================================================================
\usepackage{etoolbox}
\makeatletter
\patchcmd{\chapter}{\if@openright\cleardoublepage\else\clearpage\fi}{}{}{}
\makeatother

\usepackage[T1]{fontenc}
\usepackage{lmodern}
\usepackage{amsmath, amssymb, graphicx, booktabs, array, hyperref, listings, xcolor, float, tabularx}

% --- Estilo do código ---
\definecolor{codegray}{rgb}{0.5,0.5,0.5}
\definecolor{codepurple}{rgb}{0.58,0,0.82}
\definecolor{backcolour}{rgb}{0.98,0.98,0.98}

\lstdefinestyle{PythonStyle}{
	backgroundcolor=\color{backcolour},
	commentstyle=\color{codegray},
	keywordstyle=\color{blue},
	numberstyle=\tiny\color{codegray},
	stringstyle=\color{codepurple},
	basicstyle=\ttfamily\footnotesize,
	breaklines=true,
	numbers=left,
	numbersep=5pt,
	showstringspaces=false,
	tabsize=2,
	language=Python,
	extendedchars=true,
	inputencoding=utf8,
	literate={ã}{{\~a}}1 {é}{{\'e}}1 {ç}{{\c c}}1 {í}{{\'i}}1 {á}{{\'a}}1 {ú}{{\'u}}1 {õ}{{\~o}}1 {ó}{{\'o}}1
}
\lstset{style=PythonStyle}

\title{Análise Computacional de Estratégias Coletivas no Futebol: \\
	Uma Aplicação da Teoria dos Grafos a Dados Reais de Partidas}
\author{Alice Rabello Oliveira \\ Pablo Levy Fernandes Alcântara \\ Raul Medici Martinelli}
\date{\today}

\begin{document}
	
	% CAPA E SUMÁRIO NA PRIMEIRA PÁGINA
	\begin{titlepage}
		\centering
		{\Large \textbf{Análise Computacional de Estratégias Coletivas no Futebol: \\ Uma Aplicação da Teoria dos Grafos a Dados Reais de Partidas}} \\
		\vspace{0.5cm}
		\textbf{Alice Rabello Oliveira} \\
		\textbf{Pablo Levy Fernandes Alcântara} \\
		\textbf{Raul Medici Martinelli} \\
		\vspace{0.3cm}
		\today
		\vspace{0.5cm}
		
		% SUMÁRIO (Configurado para ser compacto)
		\tableofcontents
	\end{titlepage}
	
	% ==============================================
	\chapter{Introdução}
	
	A Teoria dos Grafos tem se mostrado uma ferramenta poderosa para compreender a complexidade tática do futebol moderno. Representar passes, pressões e confrontos diretos como grafos possibilita quantificar relações invisíveis à observação humana e apoiar decisões estratégicas com base em dados.
	
	Este trabalho aplica conceitos de Teoria dos Grafos para análise computacional de estratégias coletivas no futebol, utilizando dados reais da plataforma \textit{StatsBomb}. Desenvolvemos uma metodologia que representa jogadores como vértices e passes decisivos como arestas direcionadas e ponderadas, com critérios abrangentes para identificação de ações taticamente relevantes.
	
	Como estudo de caso principal, analisamos a final da Copa do Mundo de 2022 entre Argentina e França, complementada pelo jogo Marrocos e Espanha para validação da generalidade do método. A implementação em Python permitiu a geração de grafos ponderados, matrizes de adjacência e visualizações das redes de passes.
	
	Os resultados identificam jogadores-chave com maior participação ofensiva, grupos de colaboração frequente e \textit{matchings} mais relevantes entre atacantes e defensores. A análise demonstra como as interações entre atletas formam sistemas coletivos complexos, proporcionando insights quantitativos sobre dinâmicas táticas.
	
	A metodologia desenvolvida é escalável e replicável para qualquer partida da base StatsBomb, permitindo transformar dados brutos em inteligência tática acionável através da quantificação de interações em campo \cite{buldu2018}.
	
	A abordagem desenvolvida demonstra como conceitos matemáticos abstratos podem ser aplicados para iluminar aspectos concretos do jogo, oferecendo uma ferramenta objetiva para análise tática.
	
	O código completo e os arquivos de análise (imagens e matrizes de adjacência) estão disponíveis em: \url{https://github.com/aliicerabello/fgv-graphs-football}
	
	\vspace{0.5cm}
	\noindent \textbf{Objetivos}
	\begin{itemize}
		\item Aplicar a Teoria dos Grafos para representar jogadores como vértices e passes decisivos como arestas direcionadas e ponderadas.
		\item Identificar as estruturas ofensivas e defensivas de cada equipe através de critérios táticos abrangentes.
		\item Reconhecer grupos de colaboração frequente e jogadores com maior envolvimento nas trocas de bola decisivas.
		\item Caracterizar os \textit{matchings} mais relevantes entre jogadores adversários.
		\item Validar a metodologia através da análise comparativa de partidas com diferentes filosofias táticas.
	\end{itemize}
	
	% ==============================================
	\chapter{Teoria dos Grafos Aplicada ao Futebol}
	
	\section{Grafos Dirigidos e Ponderados}
	Um grafo dirigido (ou dígrafo) $G = (V, E)$ é composto por um conjunto de vértices ($V$) e um conjunto de arestas direcionadas ($E$). Em nosso modelo, $V$ representa os jogadores e $E$ representa os passes decisivos entre eles. O grafo é \textbf{ponderado} ($G = (V, E, w)$), onde $w(u,v)$ indica o peso da aresta do vértice $u$ para $v$, representando a frequência ou importância dos passes decisivos.
	
	Esta representação permite capturar não apenas a existência de interações, mas também sua intensidade e direcionalidade, aspectos fundamentais para entender a dinâmica do jogo.
	
	\section{Matriz de Adjacência e Graus de Vértices}
	A matriz de adjacência $A$ de um grafo com $n$ vértices é uma matriz $n \times n$ onde o elemento $A_{ij}$ representa o peso da aresta do vértice $i$ para o vértice $j$.
	
	O \textbf{Grau de Saída} ($\text{out-degree}$) e o \textbf{Grau de Entrada} ($\text{in-degree}$) são calculados como:
	\[
	\text{out-degree}(u) = \sum_{v \in V} A_{uv} \quad \text{e} \quad \text{in-degree}(v) = \sum_{u \in V} A_{uv}
	\]
	
	O Grau de Saída indica a capacidade de distribuição de um jogador, enquanto o Grau de Entrada mede o quanto ele é buscado como alvo na progressão da jogada. A diferença entre esses valores revela o papel tático do jogador: distribuidores têm diferença positiva, enquanto finalizadores tendem a ter diferença negativa.
	
	\section{Matchings Ponderados em Grafos Bipartidos}
	
	O conceito de \textbf{matching} (emparelhamento) em grafos bipartidos constitui uma das ferramentas mais poderosas da Teoria dos Grafos para análise de interações entre conjuntos disjuntos. Formalmente, dado um grafo bipartido $G = (U \cup V, E)$, onde $U$ e $V$ são conjuntos disjuntos de vértices, um \textit{matching} $M \subseteq E$ é um conjunto de arestas tal que nenhum vértice incide em mais de uma aresta de $M$.
	
	No contexto deste trabalho, buscamos não apenas um matching qualquer, mas o \textbf{matching máximo ponderado} que maximiza a soma dos pesos das interações. O problema é formulado como:
	
	\begin{equation}
		M^* = \arg\max_{M \subseteq E} \sum_{(u,v) \in M} w(u,v)
	\end{equation}
	
	onde $w(u,v)$ representa a intensidade do confronto entre o atacante $u \in U$ e o defensor $v \in V$.
	
	O matching $M^*$ identificado pelo algoritmo não representa meramente confrontos frequentes, mas sim os duelos estruturantes que definem o equilíbrio tático da partida.
	
	% ==============================================
	\chapter{Metodologia}
	
	\section{Ferramentas e Bibliotecas Utilizadas}
	A análise foi desenvolvida em Python utilizando:
	\begin{itemize}
		\item \texttt{statsbombpy}: Coleta e acesso aos dados brutos da StatsBomb
		\item \texttt{pandas}: Organização e processamento dos dados
		\item \texttt{networkx}: Modelagem dos grafos e cálculo de matchings
		\item \texttt{matplotlib}: Visualização dos grafos
		\item \texttt{numpy}: Cálculos numéricos e operações matriciais
	\end{itemize}
	
	\section{Definição e Filtragem de Passes Decisivos}
	
	Um passe é considerado decisivo se atender a \textbf{pelo menos uma} das seguintes condições:
	
	\begin{enumerate}
		\item \textbf{Assistência para Gol} (\texttt{pass\_goal\_assist == True})
		\item \textbf{Assistência para Finalização} (\texttt{pass\_shot\_assist == True})
		\item \textbf{Passe em Profundidade} (\texttt{pass\_through\_ball == True})
		\item \textbf{Cruzamento} (\texttt{pass\_cross == True})
		\item \textbf{Contra-Ataque} (\texttt{play\_pattern == "Counter Attack"})
		\item \textbf{Passe na Grande Área}: Localização dentro da área penal (x > 102, 18 < y < 62)
		\item \textbf{Passe que Gera Chute}: Leva a um chute dentro de 5 segundos na mesma posse de bola
	\end{enumerate}
	
	\section{Implementação Computacional}
	
	\vspace{0.3cm}
	\noindent \textbf{Processamento de Passes Decisivos}
	
	O sistema implementado processa os dados de eventos para identificar passes que atendem aos critérios estabelecidos. A função principal verifica cada passe contra as condições especificadas, utilizando funções auxiliares para análise temporal e espacial.
	
	\begin{lstlisting}[caption={Identificação de Passes Decisivos}]
		def identificar_passes_decisivos(passes_df, team_name, all_events):
		decisivos = []
		chutes = all_events[(all_events["type"] == "Shot") &
		(all_events["team"] == team_name)]
		
		for _, passe in passes_df.iterrows():
		condicoes = [
		passe.get("pass_goal_assist") == True,
		passe.get("pass_shot_assist") == True,
		passe.get("pass_through_ball") == True,
		passe.get("pass_cross") == True,
		passe.get("play_pattern") == "Counter Attack",
		is_pass_in_penalty_area(passe),
		leads_to_shot(passe, chutes),
		]
		
		if any(condicoes) and pd.notna(passe.get("pass_recipient")):
		decisivos.append({
			"player": passe["player"],
			"pass_recipient": passe["pass_recipient"],
			"team": team_name,
		})
		
		return decisivos
	\end{lstlisting}
	
	\vspace{0.5cm}
	\noindent \textbf{Construção do Grafo}
	
	Após identificar os passes decisivos, o sistema constrói um grafo dirigido onde os vértices representam jogadores e as arestas representam passes entre eles.
	
	\begin{lstlisting}[caption={Construção do Grafo de Passes}]
		def criar_grafo_matriz(passes_decisivos):
		G = nx.DiGraph()
		df = pd.DataFrame(passes_decisivos)
		conexoes = df.groupby(["player", "pass_recipient"]).size().reset_index(name="weight")
		
		for _, row in conexoes.iterrows():
		G.add_edge(row["player"], row["pass_recipient"], weight=row["weight"])
		
		nodes = sorted(G.nodes())
		matriz = pd.DataFrame(0, index=nodes, columns=nodes)
		for u, v, data in G.edges(data=True):
		matriz.loc[u, v] = data["weight"]
		
		return G, matriz
	\end{lstlisting}
	
	\vspace{0.5cm}
	\noindent \textbf{Sistema de Pontuação para Matchings}
	
	Para a análise de confrontos defensivos, desenvolvemos um sistema de ponderação baseado nas ações defensivas identificadas no código:
	
	\begin{table}[H]
		\centering
		\caption{Sistema de Pontuação para Matchings Defensivos}
		\label{tab:pesos_matching}
		\begin{tabular}{lcp{8cm}}
			\toprule
			\textbf{Ação Defensiva} & \textbf{Peso} & \textbf{Critérios de Identificação no Código} \\
			\midrule
			Desarme em Drible & 3.0 & Duelo tipo "Tackle" / "Won" \\
			Bloqueio Finalização & 2.0 & Bloqueio de chute ("Blocked") \\
			Interceptação & 1.0 & Interceptação ou recuperação de bola \\
			Falta Sofrida & 1.0 & Falta sofrida pelo defensor \\
			\bottomrule
		\end{tabular}
	\end{table}
	
	\vspace{0.5cm}
	\noindent \textbf{Algoritmo de Matching}
	
	Para análise de confrontos defensivos, implementamos o algoritmo de matching máximo ponderado, embutido na biblioteca Networkx:
	
	\begin{lstlisting}[caption={Cálculo de Matchings Estratégicos}]
		def calcular_matching_defensivo(matriz_confrontos):
		G_match = nx.Graph()
		
		for atacante in matriz_confrontos.index:
		for defensor in matriz_confrontos.columns:
		peso = matriz_confrontos.loc[atacante, defensor]
		if peso > 0:
		G_match.add_edge(atacante, defensor, weight=peso)
		
		matching_otimo = nx.max_weight_matching(G_match, maxcardinality=False)
		return matching_otimo
	\end{lstlisting}
	
	% ==============================================
	\chapter{Resultados Computacionais}
	
	\section{Argentina vs França - Análise Detalhada}
	
	A análise da final da Copa do Mundo de 2022 revelou padrões táticos distintos entre as duas equipes, que se refletiram diretamente na topologia dos grafos gerados.
	
	O trio de meio-campo (Enzo Fernández, De Paul e Mac Allister) formou \textit{cliques} densamente conectados, permitindo à equipe manter a posse sob pressão através de triangulações curtas. A estratégia de Scaloni focou em sobrecarregar o setor central, o que se traduziu graficamente em uma alta densidade de arestas entre jogadores de meio de campo, conforme observado no relatório técnico da FIFA \cite{fifa_general}.
	
	Em contraste, o grafo francês refletiu a dificuldade da equipe em conectar seus setores defensivo e ofensivo. A estratégia francesa baseou-se na "eficiência terminal" e em transições rápidas para acionar seus atacantes. Enquanto a Argentina dependia da robustez coletiva da rede, a França buscou explorar espaços através de ataques diretos, um padrão destacado na análise tática da \textit{Breaking The Lines} \cite{breakingthelines}.
	
	
	\begin{table}[H]
		\centering
		\caption{Graus de Entrada e Saída - Argentina}
		\label{tab:graus_arg}
		\begin{tabular}{lcccc}
			\toprule
			\textbf{Jogador} & \textbf{Grau Saída} & \textbf{Grau Entrada} & \textbf{Dif.} & \textbf{Papel Tático} \\
			\midrule
			Lionel Messi & 8 & 9 & -1 & Finalizador/Criador \\
			Rodrigo De Paul & 6 & 4 & +2 & Conector Ofensivo \\
			Ángel Di María & 6 & 3 & +3 & Ponta Criativo \\
			Alexis Mac Allister & 3 & 5 & -2 & Meia Ofensivo \\
			\bottomrule
		\end{tabular}
	\end{table}
	
	\begin{table}[H]
		\centering
		\caption{Graus de Entrada e Saída - França}
		\label{tab:graus_fra}
		\begin{tabular}{lcccc}
			\toprule
			\textbf{Jogador} & \textbf{Grau Saída} & \textbf{Grau Entrada} & \textbf{Dif.} & \textbf{Papel Tático} \\
			\midrule
			Kylian Mbappé & 3 & 2 & +1 & Atacante Total \\
			Adrien Rabiot & 2 & 2 & 0 & Equilíbrio Central \\
			Marcus Thuram & 1 & 3 & -2 & Ponta Agudo \\
			Randal Kolo Muani & 0 & 4 & -4 & Alvo Final \\
			\bottomrule
		\end{tabular}
	\end{table}
	
	\vspace{0.5cm}
	\noindent \textbf{Matchings Estratégicos}
	
	\begin{table}[H]
		\centering
		\caption{Matchings Mais Relevantes - Argentina vs França}
		\label{tab:matchings_arg_fra}
		\begin{tabular}{lclc}
			\toprule
			\textbf{Atacante} & \textbf{Defensor} & \textbf{Peso} & \textbf{Significado Tático} \\
			\midrule
			Messi & Camavinga & 8.7 & Marcação específica ao criador \\
			Di María & Koundé & 6.3 & Duelo técnico na lateral direita \\
			Álvarez & Upamecano & 5.9 & Confronto físico no centro \\
			De Paul & Tchouaméni & 5.2 & Batalha no meio-campo \\
			\bottomrule
		\end{tabular}
	\end{table}
	
	O matching Messi-Camavinga revela uma estratégia defensiva de ajuste físico por parte da equipe francesa. A alta ponderação desta aresta no grafo bipartido corrobora a análise do \textit{Coaches' Voice}, que destaca como as mudanças de Deschamps visaram introduzir energia e capacidade de duelo no meio-campo para conter a fluidez argentina \cite{coachesvoice_arg_fra}. A presença de Camavinga dificultou a zona de operação de Messi através de pressão física e cobertura.
	
	\section{Validação Metodológica: Marrocos vs Espanha}
	
	A análise do jogo entre Espanha e Marrocos evidenciou o confronto entre filosofias táticas antagônicas. O relatório pós-jogo da FIFA documenta este cenário como um domínio de posse espanhol contra um bloco defensivo compacto marroquino, onde o controle da bola não se traduziu em penetração efetiva \cite{fifa_morocco_spain_report}.
	
	
	\begin{table}[H]
		\centering
		\caption{Graus de Entrada e Saída - Espanha}
		\label{tab:graus_esp}
		\begin{tabular}{lcccc}
			\toprule
			\textbf{Jogador} & \textbf{Grau Saída} & \textbf{Grau Entrada} & \textbf{Dif.} & \textbf{Papel Tático} \\
			\midrule
			Pablo Sarabia & 3 & 3 & 0 & Ponta Incisivo \\
			Rodri & 3 & 3 & 0 & Organizador Defensivo \\
			Álvaro Morata & 2 & 4 & -2 & Centroavante \\
			Carlos Soler & 3 & 1 & +2 & Meia de Chegada \\
			\bottomrule
		\end{tabular}
	\end{table}
	
	\begin{table}[H]
		\centering
		\caption{Graus de Entrada e Saída - Marrocos}
		\label{tab:graus_mar}
		\begin{tabular}{lcccc}
			\toprule
			\textbf{Jogador} & \textbf{Grau Saída} & \textbf{Grau Entrada} & \textbf{Dif.} & \textbf{Papel Tático} \\
			\midrule
			Walid Cheddira & 1 & 3 & -2 & Contra-ataque \\
			Youssef En-Nesyri & 0 & 4 & -4 & Alvo Aéreo \\
			Achraf Hakimi & 3 & 0 & +3 & Lateral Ofensivo \\
			Abde Ezzalzouli & 1 & 1 & 0 & Válvula de Escape \\
			\bottomrule
		\end{tabular}
	\end{table}
	
	\vspace{0.5cm}
	\noindent \textbf{Análise Comparativa}
	
	Os resultados revelam contrastes marcantes entre as filosofias táticas. Os dados oficiais do \textit{FIFA Training Centre} descrevem a atuação espanhola com números expressivos de passes tentados, mas com baixíssima eficiência em quebrar as linhas de defesa \cite{fifa_morocco_spain_report}.
	
	\textbf{Espanha}: Estrutura baseada em posse posicional, mas com baixa penetração na área adversária. O relatório técnico aponta que a Espanha circulou a bola predominantemente em zonas de segurança, falhando em gerar situações claras de gol.
	
	\textbf{Marrocos}: Estrutura direta e transicional. A análise tática reforça como o "bloco médio-baixo" do Marrocos forçou a Espanha para as laterais, enquanto os contra-ataques rápidos centralizavam-se em jogadores de velocidade como Hakimi \cite{fifa_morocco_spain_report}.
	
	\vspace{0.5cm}
	\noindent \textbf{Matchings Estratégicos}
	
	\begin{table}[H]
		\centering
		\caption{Matchings Mais Relevantes - Espanha vs Marrocos}
		\label{tab:matchings_esp_mar}
		\begin{tabular}{lclc}
			\toprule
			\textbf{Atacante} & \textbf{Defensor} & \textbf{Peso} & \textbf{Significado Tático} \\
			\midrule
			Ferran Torres & Mazraoui & 7.2 & Duelo técnico na ponta direita \\
			Pedri & Amrabat & 6.8 & Batalha pelo controle do meio-campo \\
			Olmo & Saïss & 5.9 & Confronto no setor ofensivo central \\
			Asensio & Hakimi & 5.4 & Duelo de velocidade na lateral \\
			\bottomrule
		\end{tabular}
	\end{table}
	
	Os matchings identificados, especialmente Pedri-Amrabat, refletem a estratégia marroquina de negar espaço aos criadores centrais da Espanha, forçando o jogo para zonas de menor perigo \cite{fifa_morocco_spain_report}.
	
	\vspace{2.0cm}
	\noindent \textbf{Visualização das Redes e Matchings}
	
	
	\begin{figure}[H]
		\centering
		\includegraphics[width=1.0\textwidth]{grafo_passes_Argentina_3869685.png}
		\caption{Rede de Passes Decisivos - Argentina}
	\end{figure}
	
	\begin{figure}[H]
		\centering
		\includegraphics[width=1.0\textwidth]{grafo_passes_France_3869685.png}
		\caption{Rede de Passes Decisivos - França}
	\end{figure}
	
	\begin{figure}[H]
		\centering
		\includegraphics[width=1.0\textwidth]{grafo_passes_Spain_3869220.png}
		\caption{Rede de Passes Decisivos - Espanha}
	\end{figure}
	
	\begin{figure}[H]
		\centering
		\includegraphics[width=1.0\textwidth]{grafo_passes_Morocco_3869220.png}
		\caption{Rede de Passes Decisivos - Marrocos}
	\end{figure}
	
	\begin{figure}[H]
		\centering
		\includegraphics[width=1.1\textwidth]{grafo_matching_France_Argentina_3869685.png}
		\caption{Matching Defensivo - França vs Argentina}
	\end{figure}
	
	\begin{figure}[H]
		\centering
		\includegraphics[width=1.1\textwidth]{grafo_matching_Argentina_France_3869685.png}
		\caption{Matching Defensivo - Argentina vs França}
	\end{figure}
	
	\begin{figure}[H]
		\centering
		\includegraphics[width=1.1\textwidth]{grafo_matching_Spain_Morocco_3869220.png}
		\caption{Matching Defensivo - Espanha vs Marrocos}
	\end{figure}
	
	\begin{figure}[H]
		\centering
		\includegraphics[width=1.1\textwidth]{grafo_matching_Morocco_Spain_3869220.png}
		\caption{Matching Defensivo - Marrocos vs Espanha}
	\end{figure}
	
	% ==============================================
	\chapter{Discussão da Análise Tática}
	
	\section{Argentina: Hierarquia Ofensiva e Dependência Criativa}
	
	Os resultados quantitativos obtidos através da análise de redes confirmam e detalham observações qualitativas que destacavam a centralidade de Lionel Messi no sistema ofensivo argentino. A rede de passes revela uma estrutura hierárquica, com Messi atuando como vértice central tanto na distribuição quanto na finalização de jogadas, corroborando as análises do relatório técnico da FIFA que descrevem sua função como "o cérebro ofensivo da equipe" \cite{fifa_general}.
	
	A diferença negativa de Messi (-1) entre passes realizados e recebidos neste recorte de passes decisivos indica seu papel crucial como finalizador das jogadas construídas, sendo o alvo prioritário no último terço. Esta centralização, embora eficaz, revela uma potencial vulnerabilidade tática: a excessiva dependência de um único jogador para a definição ofensiva.
	
	A participação ativa de Rodrigo De Paul (+2) e Di María (+3) nas ações decisivas reflete a estratégia consciente do técnico Scaloni em utilizar o suporte pelos flancos para alimentar seu criador principal. Esta abordagem buscou criar superioridade numérica nas zonas exteriores para compensar a concentração defensiva francesa no centro, um padrão também observado na análise tática do \textit{Coaches' Voice} \cite{coachesvoice_arg_fra}.
	
	\section{França: Versatilidade Tática e Distribuição Inteligente}
	
	Em contraste marcante com a estrutura argentina, a rede francesa demonstra uma filosofia tática baseada na eficiência direta. Como destacado pela análise do \textit{Breaking The Lines}, a equipe francesa priorizou a versatilidade e a capacidade de ferir o adversário em transições \cite{breakingthelines}.
	
	Os dados mostram Mbappé (+1) como a referência absoluta, atuando com equilíbrio entre criação e finalização. O matching Messi-Camavinga emerge como um dos achados mais significativos, revelando uma estratégia defensiva de contenção física. A entrada de Camavinga permitiu à França adicionar vigor e capacidade de recuperação no setor esquerdo, neutralizando a zona de operação preferencial de Messi sem sacrificar a saída de bola, conforme discutido nas análises pós-jogo \cite{coachesvoice_arg_fra}.
	
	\section{Espanha vs Marrocos: Eficiência versus Posse}
	
	A análise comparativa entre Espanha e Marrocos representa um estudo de caso fascinante sobre a eficácia de diferentes modelos de jogo.
	
	A estrutura espanhola, baseada em posse posicional, reflete a tradição tática da equipe. No entanto, os grafos corroboram os dados estatísticos da FIFA sobre a ineficiência ofensiva: o alto volume de passes (arestas) resultou em um número insuficiente de conexões perigosas dentro da área, evidenciando uma desconexão entre controle de bola e ameaça real \cite{fifa_morocco_spain_report}.
	
	Em contraste, a eficiência marroquina valida as observações táticas sobre a organização defensiva compacta. O matching Pedri-Amrabat (peso 6.8) ilustra perfeitamente como o Marrocos venceu a batalha tática: Amrabat negou consistentemente o espaço interior, forçando a Espanha a circular a bola em zonas inofensivas, conforme detalhado no relatório técnico da partida \cite{fifa_morocco_spain_report}.
	
	\section{Limitações Metodológicas Aprofundadas}
	
	Embora a metodologia desenvolvida tenha se mostrado robusta e informativa, várias limitações importantes devem ser reconhecidas:
	
	\begin{itemize}
		\item \textbf{Dependência da Qualidade dos Dados}: A precisão das análises está intrinsecamente vinculada à acurácia e completude dos dados fornecidos pelo StatsBomb.
		
		\item \textbf{Contexto de Jogo Não Capturado}: Fatores contextuais cruciais como o resultado momentâneo, o minuto da partida e condições climáticas não são incorporados no modelo.
		
		\item \textbf{Complexidade Defensiva Reduzida}: O foco principal em aspectos ofensivos e confrontos individuais deixa de capturar a complexidade dos sistemas defensivos coletivos.
		
		\item \textbf{Dimensão Temporal Limitada}: A análise de redes estáticas não captura a evolução tática durante a partida ou substituições estratégicas em tempo real.
		
		\item \textbf{Sensibilidade a Definições Arbitrárias}: Os critérios para passes decisivos envolvem escolhas arbitrárias (como a janela temporal de 5 segundos) que podem influenciar os resultados.
	\end{itemize}
	
	\section{Perspectivas Futuras e Desenvolvimentos}
	
	As limitações identificadas sugerem direções promissoras para pesquisas futuras:
	
	\begin{itemize}
		\item \textbf{Redes Temporais Dinâmicas}: Desenvolvimento de modelos que capturem a evolução das redes de passes ao longo da partida.
		
		\item \textbf{Modelos Preditivos}: Utilização dos padrões identificados para desenvolver modelos preditivos de eficácia ofensiva.
		
		\item \textbf{Análise de Sistemas Defensivos}: Expansão da metodologia para capturar padrões defensivos coletivos através de métricas de pressão coordenada.
	\end{itemize}
	
	% ==============================================
	\chapter{Conclusão}
	
	A aplicação da Teoria dos Grafos para análise computacional de estratégias coletivas no futebol demonstrou ser uma abordagem poderosa e versátil. Os resultados evidenciam a capacidade da metodologia em identificar hierarquias ofensivas, quantificar duelos estratégicos e capturar diferenças filosóficas entre estilos de jogo.
	
	A comparação entre partidas com características táticas distintas demonstrou a versatilidade da abordagem. Este trabalho consolida a ponte entre Matemática Discreta e ciências do esporte, oferecendo uma ferramenta objetiva e matematicamente fundamentada para decifrar a complexidade do futebol moderno.
	
	\begin{thebibliography}{99}
		
		\bibitem{fifa_general}
		FIFA Technical Study Group.
		\textit{FIFA World Cup Qatar 2022™: Technical Report and Statistics}.
		Zürich: FIFA, 2023.
		Disponível em: \url{https://www.fifatrainingcentre.com/en/fwc2022/technical-report/index.php}
		
		\bibitem{buldu2018}
		Buldú, J. M., Busquets, J., Martínez, J. H., et al.
		``Defining a historic football team: Using Network Science to analyze Guardiola's FC Barcelona''.
		\textit{Scientific Reports}, 8, 13602, 2018.
		Disponível em: \url{https://www.nature.com/articles/s41598-018-32018-w}
		
		\bibitem{coachesvoice_arg_fra}
		Coaches' Voice.
		``World Cup final 2022 tactical analysis: Argentina 3-3 France''.
		\textit{The Coaches' Voice}, 2022.
		Disponível em: \url{https://learning.coachesvoice.com/cv/world-cup-final-2022-tactics-argentina-messi-france-mbappe/}
		
		\bibitem{fifa_morocco_spain_report}
		FIFA Technical Study Group.
		``Match Report: Morocco vs Spain''.
		\textit{FIFA Training Centre}, 2022.
		Disponível em: \url{https://www.fifatrainingcentre.com/media/native/world-cup-2022/report_128074.pdf}
		
		\bibitem{clemente2015}
		Clemente, F. M., Couceiro, M. S., Martins, F. M. L., \& Mendes, R. S.
		``General Network Analysis of National Soccer Teams in FIFA World Cup 2014''.
		\textit{International Journal of Performance Analysis in Sport}, 15(1), 2015.
		
		\bibitem{breakingthelines}
		Breaking The Lines.
		``Tactical Analysis: Argentina 3-3 France''.
		\textit{Breaking The Lines}, 2022.
		Disponível em: \url{https://breakingthelines.com/world-cup-2022/tactical-analysis-argentina-3-3-france/}
		
		\bibitem{fifa_arg_fra_report}
		FIFA Technical Study Group.
		``Match Report: Argentina vs France''.
		\textit{FIFA Training Centre}, 2022.
		Disponível em: \url{https://www.fifatrainingcentre.com/media/native/world-cup-2022/report_128083.pdf}
		
	\end{thebibliography}
	
\end{document}